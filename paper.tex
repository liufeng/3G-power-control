% Created 2010-02-12 Fri 12:27
\documentclass[11pt]{article}
\usepackage[utf8]{inputenc}
\usepackage[T1]{fontenc}
\usepackage{graphicx}
\usepackage{longtable}
\usepackage{float}
\usepackage{wrapfig}
\usepackage{soul}
\usepackage{amssymb}
\usepackage{hyperref}
\usepackage[top=1in, bottom=1in, left=1in, right=1in]{geometry}
\usepackage{indentfirst}

\title{paper}
\author{Feng Liu}
\date{12 February 2010}

\begin{document}

%\maketitle

%\setcounter{tocdepth}{3}
%\tableofcontents
%\vspace*{1cm}
\section{Adaptive and Predictive Downlink Resource Management in Next Generation CDMA Networks}
\label{sec-1}


In the world today, users expect to be able to move from one cell Base
Station (BS) to another without dropping a call. It is widely agreed
upon that dropping an active call is worse than rejecting a new call,
and so the need for Base Stations to be able to handle both incoming
calls and hand-off calls while still maintaining Quality of Service
(QoS) has become increasingly important.

Although Time-Division Multiple Access (TDMA) and Frequency-Division
Multiple Access (FDMA) systems can maintain QoS for hand-off calls
using a Hard Guard Channel, where resources are fixed and predicatble,
Code-Division Multiple Access (CDMA) systems are limited by power and
other concerns. This means we must provide an adaptable, Soft Guard
Channel that can adjust to varying levels of traffic at any given
time, without any assumptions of traffic and mobility patterns.

This paper addresses the need for a Soft Guard Channel to prevent
dropped calls when mobile users move from one cell to another by
proposing two methods: Guard Capacity Adaption Based on Dropping
(GAD), and Guard Capacity Adaptation Based On Prediction and Dropping
(GAPD). Both schemes are found to be effective, however GAPD is more
robust to temporal traffic variations and control parameter changes
due to its dropped call prediction method.


\section{Performance Analysis of Downlink Power Control in CDMA Systems}
\label{sec-2}


Downlink Power in CDMA networks is responsible for providing good
quality signal coverage, load balancing in inter-cell areas and
minimizing necessary power transmission required in order to reduce
cochannel interference with other cells and increase system capacity.
Power control is also important to reduce noise caused by
high-transmit power levels of nearby mobiles, and preserve the
batteries of said mobiles.

This article discusses four methods of power control for CDMA based
cellular networks. Two of which us an open loop system, while the
other two maintain a closed loop system to manage power. These
algorithms are as follows:

The Distance Based Power Allocation Algorithm (DBPA), determines power
levels required for connected mobiles without the need for correction
or feedback. Typically, users at the cell boundary are going to have
greater interference and thus will require additional power.
Conversely, the Distributed Balancing Algorithm (DBA) makes use of a
received Signal to Interference Ratio (SIR) from connected mobiles to
adjust the transmission for better quality dynamically. DBA tends to
have better results due to the feedback it receives from mobiles.

Multiple Step SIR-based Power Control (MSPC) and Adaptive Step
SIR-based Power Control (ASPC) both make use of closed-loop
mechanisms. MSPC and ASPC alike will use a number of iterations to
examine mobile feedback individually and adjust power appropriately to
those moviles as needed. The difference between the two is that ASPC
makes use of the results of its previous iteration in the
determination of power management for the current iteration. This
leads to a faster tendency towards no outages.


\section{UMTS Release 99/4 Airlink Enhancement for supporting MBMS Services}
\label{sec-3}


This paper discusses the need for multicast support in mobile
networks. There are many applications, such as distance education,
emergency situation communication, battlefield environments and more.
Methods such as longer Transmission Time Interval (TTI) and Space-time
block coding based transmit diversity (STTD) techniques, which reduce
redundancy in multiple transmitted versions of a signal, to reduce the
power requirement of delivering multicast traffic over Forward Access
Channel (FACH) for Multimedia Broadcast Multicast Service (MBMS)
users.

        

\section{Reducing the Transmission Power Requirements of the Multimedia Broadcast/Multicast Service}
\label{sec-4}


MBMS is well-suited to customers paying for a service, particularly a
video feed service. The problem with the current model arises when
multicast content is delivered to a receiver that has multiple types
of subscribing clients it becomes difficult to select an appropriate
variant of the content; High quality data factors out users with
simple terminals and those who cannot afford the service, while low
quality data will not please users who are willing to pay more for a
better service.

According to this paper, the current MBMS power model can be extended
to maximize the number of users supported, without using excessive
amounts of power. This can be done by supporting the distribution of
multiple variants of the same content to heterogeneous receivers via
an extended MBMS model.


\section{Power Efficient Radio Bearer Selection in MBMS Multicast Mode}
\label{sec-5}


This paper discusses the optimization of the Universal Mobile
Telecommunications System (UMTS) and the importance of channel
selection on a Base Station to increase MBMS capacity while
maintaining a minimal transmission power to reduce cell to cell
interference. The useful channels are the Dedicated Channel (DCH), the
Forward Access Channel (FACH) and the High Speed Downlink Shared
Channel (HSDSCH). Selection an improper channel could result in a
significant capacity decrease, and so this is an important issue.


\section{Multicast Scheduling in Cellular Data Networks}
\label{sec-6}


Modern 3G mobile networks are becoming increasingly popular for
services such as video and music streaming, but the existing multicast
schemes are too rigid and unable to adapt to users with different
channel conditions. Currently, multicast techniques employ a fixed
data rate that is applicable to users at the edge of the cell
boundary, effectively limiting the bandwidth of users closer to the
base station. Although this paper primarily discusses TDMA and
CMDA2000, it suggests that multicast management can be improved by
selecting a data rate above the worst-case when there are few enough
worst-case users to warrant such an action. This would mean certain
users would be unable to receive the transmission, while the majority
would benefit from the improved data rate.


\section{Fixed/Variable Power Multicast Over Heterogeneous Fading Channels in Cellular Networks}
\label{sec-7}


Cellular multicast currently manages power based on a worst-case
scenario. Data rate is optimized for users at the edge of the cell,
who will not be able to handle the same rates that users close to the
base station will. This paper discusses methods to achieve a better
minimum goodput across all receivers using adaptive power and rate
control through time-sharing based rate adaption with fixed-power or
variable-power control strategies. These strategies will make use of
the Lagrangian duality theory to obtain the optimal power-rate
adaption policy.


\section{A Cell-based Call Admission Control and Bandwidth Reservation Scheme for QoS Support in Wireless Cellular Networks}
\label{sec-8}


This paper discusses similar issues to the paper entitled, “Adaptive
and Predictive Downlink Resource Management in Next Generation CDMA
Networks”. Cellular base stations have a limited amount of bandwidth
that must be shared between new calls and hand-off calls when a mobile
user moves across cell boundaries onto a new base station. Methods in
use today reserve a fixed amount of bandwidth for hand-off calls and
the rest is used for new calls. This is extremely inefficient and
wasteful when there are few or no hand-off calls, as it will block new
calls from being established. An adaptable method that achieves low
new call blocking probability (CBP), low hand-off call dropping
probability (CDP) and high cell bandwidth utilization (U) is proposed
by the authors.

\end{document}