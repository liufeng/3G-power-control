% Created 2010-02-12 Fri 13:37
\documentclass[11pt]{article}
\usepackage[utf8]{inputenc}
\usepackage[T1]{fontenc}
\usepackage{graphicx}
\usepackage{longtable}
\usepackage{float}
\usepackage{wrapfig}
\usepackage{soul}
\usepackage{amssymb}
\usepackage{hyperref}
\usepackage{indentfirst}
\usepackage[top=1in, bottom=1in, left=1in, right=1in]{geometry}

\begin{document}
\section{Related Literature}

\subsection{Adaptive and Predictive Downlink Resource Management in Next Generation CDMA Networks}

This paper addresses the need for a Soft Guard Channel to prevent
dropped calls when mobile users move from one cell to another by
proposing two methods: Guard Capacity Adaption Based on Dropping
(GAD), and Guard Capacity Adaptation Based On Prediction and Dropping
(GAPD). Both schemes are found to be effective, however GAPD is more
robust to temporal traffic variations and control parameter changes
due to its dropped call prediction method.

\subsection{Performance Analysis of Downlink Power Control in CDMA Systems}

This article discusses four methods of power control for CDMA based
cellular networks in order to improve signal coverage and boost signal
quality and manage power consumption at the same time. Two of which us
an open loop system, while the other two maintain a closed loop system
to manage power. These algorithms are DBPA, DBA, MSPC and ASPC.

\subsection{UMTS Release 99/4 Airlink Enhancement for supporting MBMS Services}

This paper discusses the need for multicast support in mobile
networks. Methods are discussed, such as longer Transmission Time
Interval (TTI) and Space-time block coding based transmit diversity
(STTD) techniques, which reduce redundancy in multiple transmitted
versions of a signal, to reduce the power requirements necessary.

\subsection{Reducing the Transmission Power Requirements of the Multimedia Broadcast/Multicast Service}

MBMS is well-suited to customers paying for a service, particularly a
video feed service. The problem with the current model arises when
high quality data factors out users with simple terminals and those
who cannot afford the service, while low quality data will not please
users who are willing to pay more for a better service. The current
MBMS model can be extended to support more users without using
excessive amounts of power.

\subsection{Power Efficient Radio Bearer Selection in MBMS Multicast Mode}

This paper discusses the optimization of the Universal Mobile
Telecommunications System (UMTS) and the importance of channel
selection on a Base Station to increase MBMS capacity while
maintaining a minimal transmission power to reduce cell to cell
interference. Selection an improper channel could result in a
significant capacity decrease, and so this is an important issue.

\subsection{Multicast Scheduling in Cellular Data Modern}

The existing multicast schemes for 3G mobile networks are too rigid
and unable to adapt to users with different channel conditions. This
paper suggests that multicast management can be improved by selecting
a data rate above the worst-case when there are few enough worst-case
users to warrant such an action and most of the users would benefit
from the improved date rate.

\subsection{Fixed/Variable Power Multicast Over Heterogeneous Fading
  Channels in Cellular Networks}

This paper discusses methods to achieve a better minimum goodput
across all receivers using adaptive power and rate control through
time-sharing based rate adaption with fixed-power or variable-power
control strategies. These strategies will make use of the Lagrangian
duality theory to obtain the optimal power-rate adaption policy.

\subsection{A Cell-based Call Admission Control and Bandwidth
  Reservation Scheme for QoS Support in Wireless Cellular Networks}

This paper discusses similar issues to the paper entitled, “Adaptive
and Predictive Downlink Resource Management in Next Generation CDMA
Networks”. An adaptable method that achieves low new call blocking
probability (CBP), low hand-off call dropping probability (CDP) and
high cell bandwidth utilization (U) is proposed by the authors.

\end{document}