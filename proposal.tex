\documentclass{article}
\usepackage[top=1in, bottom=1in, left=1in, right=1in]{geometry}
\usepackage{indentfirst}
\title{3G Power}
\author{Barkman, Brent M.\\ Chorney, Luis F.\\ Liu, Feng\\ Smith, Travis E.}
\begin{document}

\maketitle

\section{Introduction}

The project we have chosen to implement is 3G power control. This
project is focused on trying to find the causes of unnecessary power
consumption in CDMA based 3G networks

As we know, every electronic device uses power. CDMA based 3G networks
are no exception as they are constantly running and using power. By
being able to control the power management of these networks we can
increase usability time and decrease unnecessary power consumption.
Because unnecessary power consumption is such a widespread issue,
there needs to be something researched and implemented to decrease
this consumption. This is an important project due to the fact that
saving power will allow these networks to continuously run while
using much less power than before.

\section{Description of the Project}

This project will focus on three main goals in 3G power control.
Understanding the power model in CDMA based 3G networks such as
figuring out how to calculate the signal noise ratios at each
individual mobile user. Understanding the feedback-based power
management ideas for down-link power control, as well as other
multicast support in cellular networks papers. Implementing a
simulation model for down-link power management for multicast traffic
in CDMA based 3G networks.

\section{Related Literature}

\section{Methodology}

The major methodology for this project centers on simulation
development. Background research is needed to provide a knowledge base
for adding capabilities to an existing simulation model and as such
the major task is the actual development of these additional features
as noted in the description.

\section{Progress to Data}

With the recent formation and meeting of the group, we are at various
levels of completion regarding the readings. Regular meeting times
have been set to Wednesdays at 2:30 pm. The progression is addressed
later in the timeline section.

\section{Project Timeline}

\begin{center}
  \begin{tabular}{l|p{8cm}|p{3cm}}
    Week Of & Task(s) & Milestone \\
    \hline
    Feb. 14 - Feb. 20 & Complete readings and find any additional, relevant sources. & \\
    Feb. 21 - Feb. 27 & Preparation of slides regarding research. Meet to plan development work. & \\
    Feb. 28 - Mar. 6 & Work on simulation develop. & \\
    Mar. 7 - Mar. 13 & Work on simulation develop. & Have a partial demo for instructor. \\
    Mar. 14 - Mar. 20 & Work on simulation develop. & \\
    Mar. 21 - Mar. 27 & Finish simulation development, work on report. & Simulation development complete. \\
    Mar. 28 - Apr. 3 & Prepare presentation / report. & \\
    Apr. 4 - Apr. 10 & Work on report. & Project presentation. \\
  \end{tabular}
\end{center}

\section{Deliverables}

\begin{itemize}
\item Project presentation
\item Project report
\item Simulator with supplemented features
\end{itemize}

\section{Assumptions}

\begin{itemize}
\item Regarding timing
  \begin{itemize}
  \item Our presentation time will be in the last week of April,
    should this change the schedule will need to be amended.
  \item That there will be adequate time after classes end and exams
    start to finish the report.
  \end{itemize}
\item That the additional features are limited in scope as noted in
  the description section.

\item That the existing state of the simulator is of satisfactory
  quality and can be relied on to generally “work as described”.

\item That the work involved in adding the features noted in the
  description is of a reasonable amount both for the instructor to
  expect of the students and the students to expect to perform for a
  grade.
\end{itemize}

\end{document}

%%% Local Variables: 
%%% mode: PDFLaTeX
%%% TeX-master: t
%%% End: 
