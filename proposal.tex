\documentclass{article}
\usepackage[top=1in, bottom=1in, left=1in, right=1in]{geometry}
\usepackage{indentfirst}
\title{3G Power Control: Project Proposal}
\author{Barkman, Brent M.\\ Chorney, Luis F.\\ Liu, Feng\\ Smith, Travis E.}
\begin{document}

\bibliographystyle{amsalpha}

\maketitle

\section{Introduction}

The project we have chosen to implement is 3G power control. This
project is focused on trying to find the causes of unnecessary power
consumption in CDMA based 3G networks

As we know, every electronic device uses power. CDMA based 3G networks
are no exception as they are constantly running and using power. By
being able to control the power management of these networks we can
increase usability time and decrease unnecessary power consumption.
Because unnecessary power consumption is such a widespread issue,
there needs to be something researched and implemented to decrease
this consumption. This is an important project due to the fact that
saving power will allow these networks to continuously run while
using much less power than before.

\section{Description of the Project}

This project will focus on three main goals in 3G power control.
Understanding the power model in CDMA based 3G networks such as
figuring out how to calculate the signal noise ratios at each
individual mobile user. Understanding the feedback-based power
management ideas for down-link power control, as well as other
multicast support in cellular networks papers. Implementing a
simulation model for down-link power management for multicast traffic
in CDMA based 3G networks.

\section{Related Literature}

\subsection{Adaptive and Predictive Downlink Resource Management in
  Next Generation CDMA Networks}

This paper addresses the need for a Soft Guard Channel to prevent
dropped calls when mobile users move from one cell to another by
proposing two methods: Guard Capacity Adaption Based on Dropping
(GAD), and Guard Capacity Adaptation Based On Prediction and Dropping
(GAPD). Both schemes are found to be effective, however GAPD is more
robust to temporal traffic variations and control parameter changes
due to its dropped call prediction method.~\cite{wang2005adaptive}

\subsection{Performance Analysis of Downlink Power Control in CDMA
  Systems}

This article discusses four methods of power control for CDMA based
cellular networks in order to improve signal coverage and boost signal
quality and manage power consumption at the same time. Two of which us
an open loop system, while the other two maintain a closed loop system
to manage power. These algorithms are DBPA, DBA, MSPC and
ASPC.~\cite{das2003performance}

\subsection{UMTS Release 99/4 Airlink Enhancement for supporting MBMS
  Services}

This paper discusses the need for multicast support in mobile
networks. Methods are discussed, such as longer Transmission Time
Interval (TTI) and Space-time block coding based transmit diversity
(STTD) techniques, which reduce redundancy in multiple transmitted
versions of a signal, to reduce the power requirements
necessary.~\cite{chuah2004umts}

\subsection{Reducing the Transmission Power Requirements of the
  Multimedia Broadcast/Multicast Service}

MBMS is well-suited to customers paying for a service, particularly a
video feed service. The problem with the current model arises when
high quality data factors out users with simple terminals and those
who cannot afford the service, while low quality data will not please
users who are willing to pay more for a better service. The current
MBMS model can be extended to support more users without using
excessive amounts of power.~\cite{xylomenos2007reducing}

\subsection{Power Efficient Radio Bearer Selection in MBMS Multicast
  Mode}

This paper discusses the optimization of the Universal Mobile
Telecommunications System (UMTS) and the importance of channel
selection on a Base Station to increase MBMS capacity while
maintaining a minimal transmission power to reduce cell to cell
interference. Selection an improper channel could result in a
significant capacity decrease, and so this is an important
issue.~\cite{alexiou2007power}

\subsection{Multicast Scheduling in Cellular Data Networks}

The existing multicast schemes for 3G mobile networks are too rigid
and unable to adapt to users with different channel conditions. This
paper suggests that multicast management can be improved by selecting
a data rate above the worst-case when there are few enough worst-case
users to warrant such an action and most of the users would benefit
from the improved date rate.~\cite{won2007multicast}

\subsection{Fixed/Variable Power Multicast Over Heterogeneous Fading
  Channels in Cellular Networks}

This paper discusses methods to achieve a better minimum good-put
across all receivers using adaptive power and rate control through
time-sharing based rate adaption with fixed-power or variable-power
control strategies. These strategies will make use of the Lagrangian
duality theory to obtain the optimal power-rate adaption
policy.~\cite{du-fixed}

\subsection{A Cell-based Call Admission Control and Bandwidth
  Reservation Scheme for QoS Support in Wireless Cellular Networks}

This paper discusses similar issues to the paper entitled, “Adaptive
and Predictive Downlink Resource Management in Next Generation CDMA
Networks”. An adaptable method that achieves low new call blocking
probability (CBP), low hand-off call dropping probability (CDP) and
high cell bandwidth utilization (U) is proposed by the
authors.~\cite{liu2005}

\section{Methodology}

The major methodology for this project centers on simulation
development. Background research is needed to provide a knowledge base
for adding capabilities to an existing simulation model and as such
the major task is the actual development of these additional features
as noted in the description.

\section{Progress to Data}

With the recent formation and meeting of the group, we are at various
levels of completion regarding the readings. Regular meeting times
have been set to Wednesdays at 2:30 pm. The progression is addressed
later in the timeline section.

\section{Project Timeline}

\begin{center}
  \begin{tabular}{l|p{8cm}|p{3cm}}
    Week Of & Task(s) & Milestone \\
    \hline
    Feb. 14 - Feb. 20 & Complete readings and find any additional, relevant sources. & \\
    Feb. 21 - Feb. 27 & Preparation of slides regarding research. Meet to plan development work. & \\
    Feb. 28 - Mar. 6 & Work on simulation develop. & \\
    Mar. 7 - Mar. 13 & Work on simulation develop. & Have a partial demo for instructor. \\
    Mar. 14 - Mar. 20 & Work on simulation develop. & \\
    Mar. 21 - Mar. 27 & Finish simulation development, work on report. & Simulation development complete. \\
    Mar. 28 - Apr. 3 & Prepare presentation / report. & \\
    Apr. 4 - Apr. 10 & Work on report. & Project presentation. \\
  \end{tabular}
\end{center}

\section{Deliverables}

\begin{itemize}
\item Project presentation
\item Project report
\item Simulator with supplemented features
\end{itemize}

\section{Assumptions}

\begin{itemize}
\item Regarding timing
  \begin{itemize}
  \item Our presentation time will be in the last week of April,
    should this change the schedule will need to be amended.
  \item That there will be adequate time after classes end and exams
    start to finish the report.
  \end{itemize}
\item That the additional features are limited in scope as noted in
  the description section.

\item That the existing state of the simulator is of satisfactory
  quality and can be relied on to generally “work as described”.

\item That the work involved in adding the features noted in the
  description is of a reasonable amount both for the instructor to
  expect of the students and the students to expect to perform for a
  grade.
\end{itemize}

\newpage

\bibliography{proposal}

\end{document}

%%% Local Variables: 
%%% mode: PDFLaTeX
%%% TeX-master: t
%%% End: 
